\RequirePackage{iftex}
\RequirePDFTeX
\NeedsTeXFormat{LaTeX2e}
\documentclass[10pt]{article}
\usepackage{ismll-packages}
\usepackage{ismll-mathoperators}
\usepackage{ismll-style}
\usepackage{unicode-symbols}

\title{Neural Filtering for State Space Models}
\author{Randolf Scholz}
\begin{document}

\maketitle

A filter is a function $F:𝓧⊕𝓨 ⟶ 𝓧$, that, given an observation $y∈𝓨$ and a state $x∈𝓧$ returns an updated state estimate $x'∈𝓧$
%
\begin{align*}%
x' = F(x, y)%
\end{align*}%
%
We say a filter is \emph{autoregressive}, if $𝓨=𝓧$ ($𝓨={^∙}\!𝓧$ including nan values), i.e. the observation space $𝓨$ is equal to the state space $𝓧$. In this case we usually write $x^\text{obs}$ instead of $y$.
%
We say an autoregressive filter is \emph{idempotent}, if and only if $x≡x^\text{obs} ⟹ x' = x$.\footnote{$u≡v$ if $uᵢ=vᵢ$ for all non-nan components.}
More specifically, we say a filter is continuously (differentiably) idempotent, if $F(x, x^\text{obs}) = G(x-x^\text{obs})$ for some continuous (differentiable) function $G$ with $z≡0 ⟹ G(z)=0$.
%
We also consider the problem of dealing with multiple simultaneous measurements. In this case, we ask that the filter should be order-independent, for which there are 2 options:


\begin{outline}%
\1 satisfy $F(F(x, y₁), y₂) = F(F(x, y₂), y₁)$
\1 change the filter to accept multiple y-values: $F:𝓧⊕⋃_{n∈ℕ}𝓨ⁿ ⟶ 𝓧$ such that $F(x, Y) = F(x, P⋅Y)$ for any permutation matrix $P$.
\end{outline}%

Additionally, it should satisfy a certain scale invariance:





%
\begin{example}[Linear Filter]%
\label{ex: label}%
We say a filter is \emph{linear}, if it is of the form $F(x,y) = Ax + By$. When is such a filter order-independent?
%
\begin{align*}%
F(F(x, y₁), y₂) = F(F(x, y₁), y₂)
&⟺ A(Ax+By₁) + By₂ = A(Ax+By₂) + By₁
\\&⟺ ABy₁ + By₂ = ABy₂ + By₁ %
\\&⟺ AB(y₁-y₂) = B(y₁-y₂)
\\&⟺ (𝕀-A)B(y₁-y₂) = 0
\end{align*}%
%
Since the equality must hold all combinations of $y₁$ and $y₂$, it follows that $(𝕀-A)B=0$, i.e. $B=AB$.
%
Alternatively, consider the multi-value linear filter $f(x, Y) = Ax + g(Y)$, where $g:𝓨^{⊕ℕ}⟶𝓧$ is linear. We can show such $g$ are of the form $g(Y) = 𝟏ₙ^⊤ Y B$.

\end{example}%



\begin{example}[conditional filtering]

Consider a linear control system
%
\begin{align*}%
\dot{x} &= Ax + Bu + ν \\%
y &= Cx + Du + ω
\end{align*}%
%
In this situation, we consider \emph{conditional filters} that have an additional input:
%
\begin{align*}%
F:𝓧⊕𝓨⊕𝓤 ⟶ 𝓧, (x, y, u) ⟼ x' %
\end{align*}%
%
\end{example}



\section{The Classical Kalman Filter}

The classical Kalman Filter is an analytical result that assumes a normal distribution evolving under linear dynamics. Since the family of normal distributions is closed under linear transformations, the distribution is normal for all times $t$.

%
\begin{lemma}[Normal distribution under linear ODE]%
\label{lem: linear ODE normal}%

Consider the stochastic linear system $̇x(t) = A(t)x(t) + f(t) + ν$, such that $x(t₀)∼𝓝(x∣μ₀,Σ₀)$. Then $xₜ∼𝓝(x∣μₜ,Σₜ)$ with

Technically we  need to do Ito calculus here.



%
\begin{align*}%
\dot{Σ}(t) &= AΣ(t) + Σ(t)A^⊤ + GQG^⊤ - ΣHR^{-1}H^⊤Σ(t)%
\end{align*}%
%


\begin{proof}

Without noise:

Then this simply becomes a standard Linear System. We consider several cases


\begin{enumerate}%
	\item LTI homogeneous
	\item LTI inhomogeneous
	\item LTI + LTV inhomogeneity.
	\item LTV homogeneous
	\item LTV inhomogenous
\end{enumerate}%



\begin{align*}%
math%
\end{align*}%
%
\end{proof}
%






%
\end{lemma}%








\end{document}
