\RequirePackage{iftex}
\RequirePDFTeX
\NeedsTeXFormat{LaTeX2e}
\documentclass[a4paper, 10pt, margin=2cm]{scrartcl}
\usepackage{ismll-packages}
\usepackage{ismll-mathoperators}
\usepackage{ismll-style}
\usepackage{unicode-symbols}

\usepackage[margin=2cm]{geometry}

\title{Linear Systems}
\author{Randolf Scholz}
\begin{document}

\maketitle




\section{Determininistic Linear Systems}

In the most general case we consider
%
\begin{align*}%
\dot{x} = A(t)x(t) + b(t)%
\end{align*}%
%
Sometimes, we consider, redundantly, the factorization
%
\begin{align*}%
\dot{x} = A(t)x(t) + B(t)u(t)%%
\end{align*}%
%

%
\begin{table}[H]%
\centering%
\caption{caption}%
\label{tab: label}%
\begin{tabular}{l|ll|ll}%
\toprule
& \multicolumn{2}{c|}{deterministic} & \multicolumn{2}{c}{stochastic}
\\                  & homogeneous & inhomogenenous & homogeneous & inhomogenenous
\\ \midrule
    constant-coeff. & $̇x = Ax$       & $̇x = Ax + b$       & $̇x = Ax + ν$    & $̇x = Ax + b + ν$
\\  time-invariant  & $̇x = Ax$       & $̇x = Ax + b(t)$    & $̇x = Ax + ν$    & $̇x = Ax + b(t) + ν$
\\  time-variant    & $̇x = A(t)x$    & $̇x = A(t)x + b(t)$ & $̇x = A(t)x + ν$ & $̇x = A(t)x + b(t) + ν$
\\ \bottomrule
\end{tabular}%
\end{table}%


Solutions


%
\begin{table}[H]%
\centering%
\caption{}%
\label{tab: label}%
\begin{tabular}{l|ll}%
\toprule
& \multicolumn{2}{c|}{deterministic}
\\                  & homogeneous & inhomogenenous
\\ \midrule
    constant-coeff.               & $x(t+∆t) = e^{A∆t}x(t)$       & $x(t+∆t) = e^{A∆t}x(t) + φ₁(A∆t)b$
\\  time-invariant                & $x(t+∆t) = e^{A∆t}x(t)$       & $x(t+∆t) = e^{A∆t}x(t) + ∫_0^{∆t} e^{A(∆t-∆τ)}b(t+∆τ)\dd{∆τ}$
\\  time-variant (commutative)    & $̇x(t+∆t) = e^{∫ₜ^{t+∆t}A(τ)\dd{τ}}x(t)$ & $̇x(t+∆t) = e^{∫ₜ^{t+∆t}A(τ)\dd{τ}}x(t) + ∫ₜ^{t+∆t} e^{-∫_{τ}^{t+∆t}A(s)\dd{s}}b(τ)\dd{τ}$
\\  time-variant                  & no closed form                  & no closed form
\\ \bottomrule
\end{tabular}%
\end{table}%





\subsection{Linear-Gaussian Systems}

We assume $xₜ∼𝓝(μₜ,Σₜ)$. Then what is the distribution of $x(t+∆t)$?


\begin{lemma}[Expectation/Variance under linear/affine transformations]%
\label{lem: linear transformation exepctation variance}
\begin{align*}%
𝐄[Ax + b] &= A𝐄[x]+b   & 𝐕[Ax+b] &= A𝐕[x]A^⊤
\end{align*}%
\end{lemma}%

%
\begin{lemma}[Normal under linear/affine transformation]%
\label{lem: gaussian linear}%
%
\begin{align*}%
x∼𝓝(μ,Σ) ⟹ Ax + b ∼𝓝(Aμ+b,AΣA^⊤)%
\end{align*}%
%
\begin{proof}
Via characteristic functions. Note that $φ_{AX+b}(𝐭) = e^{ib^⊤𝐭}φ_X(A^⊤𝐭)$.
then

%
\begin{align*}%
φ_y(𝐭) = 𝐄_y[e^{i 𝐲^⊤ 𝐭}] = 𝐄_x[e^{i (A𝐱+b)^⊤ 𝐭}] = 𝐄_x[e^{i (A𝐱+b)^⊤ 𝐭}] =  e^{ib^⊤ 𝐭}𝐄_x[e^{i 𝐱^⊤A^⊤𝐭}]
\\ =  e^{ib^⊤ 𝐭}  e^{iμ^⊤A^⊤𝐭 - ½𝐭AΣA^⊤𝐭} =  e^{i(Aμ+b)^⊤𝐭 - ½𝐭AΣA^⊤𝐭}%
\end{align*}%
%
\end{proof}
%
\end{lemma}%


%
\begin{lemma}[characteristic function of gaussian]%
\label{lem: label}%
%
\begin{align*}%
x∼𝓝(μ,Σ) ⟹ φ(𝐭) = 𝐄_x [e^{i x^⊤𝐭}] = e^{iμ^⊤ 𝐭 - ½𝐭^⊤Σ𝐭}%
\end{align*}%
%
%
\end{lemma}%





\section{Stochastic Linear Systems}


First consider the homogeneous case:
%
\begin{align*}%
 ̇xₜ = Aₜxₜ + νₜ%
\end{align*}%
%
where $νₜ$ is a zero-mean white noise process, i.e. $𝐄[νₜ] = 0$ and $𝐄[νₜνₛ] = δ_{ts} Qₜ$.

Then

%
\begin{align*}%
\dot{Σ} = AₜΣₜ + ΣₜAₜ^⊤ + Qₜ%
\end{align*}%
%






\end{document}
