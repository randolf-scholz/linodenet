\RequirePackage{iftex}
\RequirePDFTeX
\NeedsTeXFormat{LaTeX2e}
\documentclass[10pt]{article}
\usepackage{ismll-packages}
\usepackage{ismll-mathoperators}
\usepackage{ismll-style}
\usepackage{unicode-symbols}

\title{Derivative of the first order SVD}
\author{Randolf Scholz}
\begin{document}

\maketitle


Consider computing the first order SVD expansion. By the Eckart–Young–Mirsky theorem, this is equivalent to solving


%
\begin{align*}%
\minimize_{σ,u,v} ½‖A - σuv^⊤‖²_F \qq{s.t.}  ‖u‖=1 \qq{and} ‖v‖=1%
\end{align*}%
%
Any triplet $(σ,u,v)$ for a \textbf{unique} singular value satisfies


\paragraph{The jacobian and lagrangian}
%
The derivative of the objective function is

\begin{align*}%
𝐉_f(A,\bsmat{σ \\ u \\ v}) = \bmat{ A-σuv^⊤ &  \mat{ σ - u^⊤Av \\ σ²u - σAv \\ σ²v - σA^⊤u}}
⟹ 𝐇_f(\bsmat{σ \\ u \\ v}) = \bmat{ 1 & 2σu - Av & 2σv - A^⊤u \\ -Av & σ²𝕀ₘ  & -σA \\ -A^⊤u & -σA^⊤ & σ²𝕀ₙ }
\end{align*}%
%
Consider the function
%
\begin{align*}%
f(A,\bsmat{σ \\ u \\ v}) = \pmat{ σ - u^⊤Av \\ σ²u - σAv \\ σ²v - σA^⊤u} ≡ 𝟎%
⟹ 𝐉_f(A,\bsmat{σ \\ u \\ v}) = \barr{c|ccc}{-ξvu^⊤ & 1 & 2σu - Av & 2σv - A^⊤u \\ -σvϕ^⊤ & -Av & σ²𝕀ₘ  & -σA \\ -σuψ^⊤ & -A^⊤u & -σA^⊤ & σ²𝕀ₙ }
\end{align*}%
%
Thus, gradient descent schema is

%
\begin{align*}%
\begin{aligned}
σ' &= σ - η_σ(σ - u^⊤Av) \\
u' &= u - η_u(σ²u - σAv) \\
v' &= v - η_v(σ²v - σA^⊤u)
\end{aligned}
\end{align*}%
%
And the newton step with diagonal approximation of the hessian:
%
\begin{align*}%
\begin{aligned}
σ' &= σ - 1(σ - u^⊤Av)              &&= u^⊤Av \\
u' &= u - \tfrac{1}{σ²}(σ²u - σAv)  &&= \tfrac{1}{σ}Av \\
v' &= v - \tfrac{1}{σ²}(σ²v - σA^⊤u)&&= \tfrac{1}{σ}A^⊤u
\end{aligned}
\end{align*}%
%





%
\begin{align*}%
	σ &= u^⊤ A v
\\  Av &= σu
\\  A^⊤u &= σv%
\end{align*}%
%
from this we can derive
%
\begin{align*}%
∆σ &= {∆u}^⊤ A v + u^⊤{∆A}v + u^⊤A{∆v}
\\ &= {∆u}^⊤u + u^⊤{∆A}v + v^⊤{∆v}
\\ &= u^⊤{∆A}v%
\end{align*}%
%
Where in the last step we used $∆u⟂u$ and $∆v⟂v$, which follows from the side condition. Further we have:
%
\begin{align*}%
\begin{aligned}
   {∆A}v  + A{∆v} &= {∆σ}u + σ{∆u}
\\ {∆A}^⊤u + A^⊤{∆u} &= {∆σ}v + σ{∆v}%
\end{aligned}
\iff
\underbrace{\bmat[c]{σ𝕀ₘ & -A \\ -A^⊤ & σ𝕀ₙ}}_{≕K}⋅\bmat{∆u\\∆v} = \bmat{{∆A}v - {∆σ}u \\ {∆A}^⊤u - {∆σ}v}
\end{align*}%
%
\section{The VJP}
The last equation allows us to compute the VJP at ease:
%
\begin{align*}%
\Bigl{⟨}\bmat{ϕ\\ψ}\Bigm{\vert}\bmat{∆u\\∆v}\Bigr{⟩}
&= \Bigl{⟨}\bmat{ϕ\\ψ}\Bigm{\vert} K^{-1}\bmat{{∆A}v - {∆σ}u \\ {∆A}^⊤u - {∆σ}v}\Bigr{⟩}
\\&= \Bigl{⟨}K^{-⊤}\bmat{ϕ\\ψ}\Bigm{\vert}\bmat{{∆A}v - {∆σ}u \\ {∆A}^⊤u - {∆σ}v}\Bigr{⟩}%
\\&= \Bigl{⟨}\bmat{\tilde{ϕ}\\\tilde{ψ}}\Bigm{\vert}\bmat{{∆A}v - {∆σ}u \\ {∆A}^⊤u - {∆σ}v}\Bigr{⟩}%
\end{align*}%
%
Now, we compute the terms individually:
%
\begin{align*}%
⟨\tilde{ϕ}∣ {∆A}v - {∆σ}u⟩
&= ⟨\tilde{ϕ}v^⊤∣ {∆A}⟩ - ⟨u^⊤\tilde{ϕ}∣{∆σ}⟩%
\\&= ⟨\tilde{ϕ}v^⊤∣ {∆A}⟩ - ⟨u^⊤\tilde{ϕ}∣u^⊤{∆A}v⟩%
\\&= ⟨(𝕀ₘ - uu^⊤)\tilde{ϕ}v^⊤∣{∆A}⟩
\end{align*}%
%
And for the second term we get
%
\begin{align*}%
⟨\tilde{ψ} ∣ {∆A}^⊤ u - {∆σ}v⟩
&= ⟨\tilde{ψ}u^⊤∣ {∆A}^⊤⟩ - ⟨v^⊤\tilde{ψ}∣{∆σ}⟩%
\\&= ⟨u\tilde{ψ}^⊤∣ {∆A}⟩ - ⟨\tilde{ψ}^⊤v∣u^⊤{∆A}v⟩%
\\&= ⟨u\tilde{ψ}(𝕀ₙ - vv^⊤)∣{∆A}⟩
\end{align*}%
%
Using the formula for inverting a block-matrix, we can give an explicit solution to $K^{-⊤}$:
%
\begin{align*}%
\bmat[c]{σ𝕀ₘ & -A \\ -A^⊤ & σ𝕀ₙ}^{-1}
&= \bmat{(σ𝕀ₘ - \frac{1}{σ}AA^⊤)^{-1} & 𝟎_{m×n} \\ 𝟎_{n×m} & (σ𝕀ₙ - \frac{1}{σ}A^⊤A)^{-1}}⋅\bmat{𝕀ₘ & \frac{1}{σ}A \\ \frac{1}{σ}A^⊤ & 𝕀ₙ}%
\\&= \bmat{\tfrac{1}{σ}(𝕀ₘ - \frac{1}{σ²}AA^⊤)^{-1} & \tfrac{1}{σ²}(𝕀ₘ - \frac{1}{σ²}AA^⊤)^{-1}A \\ \frac{1}{σ²}(𝕀ₙ - \frac{1}{σ²}A^⊤A)^{-1}A^⊤ & \tfrac{1}{σ}(𝕀ₙ - \frac{1}{σ²}A^⊤A)^{-1} }
\\&= \frac{1}{σ}\bmat{(𝕀ₘ - \tilde{A}\tilde{A}^⊤)^{-1} & (𝕀ₘ - \tilde{A}\tilde{A}^⊤)^{-1}\tilde{A} \\ (𝕀ₙ - \tilde{A}^⊤\tilde{A})^{-1}\tilde{A}^⊤ & (𝕀ₙ - \tilde{A}^⊤\tilde{A})^{-1} }
\end{align*}%
%
And we see it's basically projection operators with respect to the image/kernel of $\tilde{A} = \frac{1}{σ}A$.
%
In summary, we obtain the following formula for the VJP:

%
\begin{align*}%
\bmat[c]{σ𝕀ₘ & -A \\ -A^⊤ & σ𝕀ₙ}^{-⊤}\bmat{\tilde{ϕ}\\ \tilde{ψ}} = \bmat{ϕ\\ψ}
&⟺
\bmat{\tilde{ϕ}\\ \tilde{ψ}} = \frac{1}{σ}\bmat{
		(𝕀ₘ - \tilde{A}\tilde{A}^⊤)^{-1} & (𝕀ₘ - \tilde{A}\tilde{A}^⊤)^{-1}\tilde{A}
	\\  (𝕀ₙ - \tilde{A}^⊤\tilde{A})^{-1}\tilde{A}^⊤ & (𝕀ₙ - \tilde{A}^⊤\tilde{A})^{-1}
}^⊤ \bmat{ϕ\\ψ}%
\\&⟺
\bmat{\tilde{ϕ}\\ \tilde{ψ}} = \frac{1}{σ}\bmat{
		(𝕀ₘ - \tilde{A}\tilde{A}^⊤)^{-1} & \tilde{A}(𝕀ₙ - \tilde{A}^⊤\tilde{A})^{-1}
	\\  \tilde{A}^⊤(𝕀ₘ - \tilde{A}\tilde{A}^⊤)^{-1}  & (𝕀ₙ - \tilde{A}^⊤\tilde{A})^{-1}
} \bmat{ϕ\\ψ}%
\end{align*}%
%
Thus, we need to solve 4 linear systems:
%
\begin{align*}%
(𝕀ₘ - \tilde{A}\tilde{A}^⊤)x &= ϕ
& (𝕀ₙ - \tilde{A}^⊤\tilde{A})y &= \tilde{A}ψ%
\\ (𝕀ₘ - \tilde{A}\tilde{A}^⊤)z &= \tilde{A}^⊤ϕ%
& (𝕀ₙ - \tilde{A}^⊤\tilde{A})w &= ψ
\end{align*}%
%
Then $\tilde{ϕ} = \frac{1}{σ}(x+y)$ and $\tilde{ψ} = \frac{1}{σ}(z+w)$, and the VJP are given by the previous equations:
%
\begin{align*}%
	ξ^⊤\frac{∂σ}{∂A} &= ξuv^⊤
\\  ϕ^⊤\frac{∂u}{∂A} &= (𝕀ₘ - uu^⊤)\tilde{ϕ}v^⊤%
\\  ψ^⊤\frac{∂v}{∂A} &= u\tilde{ψ}(𝕀ₙ - vv^⊤)%
\end{align*}%
%
\end{document}
